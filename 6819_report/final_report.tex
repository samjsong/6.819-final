\documentclass[10pt,twocolumn,letterpaper]{article}

\usepackage{cvpr}
\usepackage{times}
\usepackage{epsfig}
\usepackage{graphicx}
\usepackage{amsmath}
\usepackage{amssymb}

% Include other packages here, before hyperref.

% If you comment hyperref and then uncomment it, you should delete
% egpaper.aux before re-running latex.  (Or just hit 'q' on the first latex
% run, let it finish, and you should be clear).
\usepackage[breaklinks=true,bookmarks=false]{hyperref}

\cvprfinalcopy % *** Uncomment this line for the final submission

\def\cvprPaperID{****} % *** Enter the CVPR Paper ID here
\def\httilde{\mbox{\tt\raisebox{-.5ex}{\symbol{126}}}}

% Pages are numbered in submission mode, and unnumbered in camera-ready
%\ifcvprfinal\pagestyle{empty}\fi
\setcounter{page}{1}
\begin{document}

%%%%%%%%% TITLE
\title{Neural Image Style Transfer}

\author{Veronica Lee\\
MIT\\
Cambridge, MA, USA\\
{\tt\small vslee@mit.edu}
% For a paper whose authors are all at the same institution,
% omit the following lines up until the closing ``}''.
% Additional authors and addresses can be added with ``\and'',
% just like the second author.
% To save space, use either the email address or home page, not both
\and
Samuel Song\\
MIT\\
Cambridge, MA, USA\\
{\tt\small samjsong@mit.edu}
}

\maketitle
%\thispagestyle{empty}

%%%%%%%%% ABSTRACT
\begin{abstract}
   The ABSTRACT is to be in fully-justified italicized text, at the top
   of the left-hand column, below the author and affiliation
   information. Use the word ``Abstract'' as the title, in 12-point
   Times, boldface type, centered relative to the column, initially
   capitalized. The abstract is to be in 10-point, single-spaced type.
   Leave two blank lines after the Abstract, then begin the main text.
   Look at previous CVPR abstracts to get a feel for style and length.
\end{abstract}

%%%%%%%%% BODY TEXT
\section{Introduction}

A pastiche is a work of art that imitates the style of other works of art. The computer vision and machine learning technologies that exist today allow pastiches to be created digitally. Specifically, this automated process of recomposing an image to consist of styles from other images is called an image style transfer. In this paper, we will discuss the implementation of image style transfer using a deep neural network. 

%------------------------------------------------------------------------
\section{Related Work}



%------------------------------------------------------------------------
\section{Image Content and Style}

approach setup and definitions

\subsection{Content Representation and Reconstruction}

content representation

\subsection{Style Representation and Reconstruction}

style representation

%------------------------------------------------------------------------
\section{Image Style Transfer Using Neural Networks}

approach methods and implementation

\subsection{Methodology}

\subsection{Neural Algorithm}

%------------------------------------------------------------------------
\section{Results}

results

%------------------------------------------------------------------------
\section{Conclusion}

conclusion


{\small
\bibliographystyle{ieee}
\bibliography{egbib}
}

\end{document}
